%%
%
% Skia
% skia@libskia.so
%
%%

\documentclass[a4paper]{report}

%packages
\usepackage[utf8]{inputenc}
\usepackage[francais]{babel}
\usepackage{graphicx}\graphicspath{{pix/}}
\usepackage{float}
\usepackage{scrextend}
\usepackage[T1]{fontenc}
\usepackage{color}
\usepackage{fancyhdr}
%Options: Sonny, Lenny, Glenn, Conny, Rejne, Bjarne, Bjornstrup
\usepackage[Bjornstrup]{fncychap}
\usepackage{minted}
\usepackage[colorlinks=true,linkcolor=black]{hyperref}
\usepackage{pdfpages}
%\usepackage{titlesec, blindtext, color}

%pdf metadata
\hypersetup{
    unicode=true,
    colorlinks=true,
    citecolor=black,
    filecolor=black,
    linkcolor=black,
    urlcolor=black,
    pdfauthor={Skia <skia@libskia.so>},
    pdftitle={},
    pdfcreator={pdftex},
    pdfsubject={},
    pdfkeywords={},
}


\definecolor{keywords}{RGB}{200,0,90}
\definecolor{comments}{RGB}{50,50,253}
\definecolor{red}{RGB}{160,0,0}
\definecolor{brown}{RGB}{160,100,100}
\definecolor{green}{RGB}{0,200,0}
\definecolor{darkgreen}{RGB}{0,130,0}
\definecolor{gray}{RGB}{100,100,100}


%inner meta
\title{Sith: Détail de quelques applications}
\author{Skia (Florent JACQUET)\\
Lo-J (Guillaume Renaud)}
\date{Dernière version: \today}

\begin{document}

\maketitle

\tableofcontents

\chapter*{Remerciements}
\section*{Lo-J (Guillaume RENAUD)}
\par Je remercie tout d’abord Monsieur Frédéric Lassabe qui nous a permis d’effectuer cette TO52 lors de notre cursus à l’UTBM, nous permettant ainsi de mêler nôtre travail scolaire à nôtre envie de participer à l’amélioration de la vie associative de l’UTBM.

\par Je tiens aussi à remercier Florent Jacquet qui m’a aidé tout au long de ce travail et à qui j’ai pu poser mes différentes questions pour apprendre et comprendre plus rapidement que si j’avais été seul.


\chapter{Introduction}
\par Après le développement de la base du nouveau site de l'AE, le projet \emph{Sith}, au Printemps 2016, la mise en
production a pu avoir lieu avec succès fin Août 2016.

\par Mais le site était encore très incomplet, et il était nécessaire d'y ajouter un grand nombre de fonctionnalités
moins critiques, celles-ci n'ayant pas de rapport avec l'argent, mais tout de même très utiles pour le fonctionnement
de l'AE.

\chapter{Eboutic}
\label{sec:eboutic}
\par Développeur principal: Skia

\section{But}
\label{sub:but}
\par Fournir une boutique en ligne, avec paiement sécurisé, compatible avec l'API de paiement du Crédit Agricole.
\begin{itemize}
    \item Gérer les cotisations
    \item Gérer les rechargements de compte AE
    \item Gérer différents groupes de vente
\end{itemize}

\section{Principaux problèmes}
\label{sec:principaux_problemes}

\subsection{Interaction avec l'API}
\label{sub:interaction_avec_l_api}
\par C'est la principale contrainte de cette application. On doit interagir avec les serveurs du Crédit Agricole, et
pour cela, ces derniers n'aident pas beaucoup.

\par Ils fournissent un PDF peu clair\footnote{disponible dans le dossier \url{doc/Etransaction/} des sources du site}
expliquant l'implémentation d'un site marchand, en plus des nombreux autres PDF de documentation disponibles à l'adresse
\url{https://e-transactions.avem-groupe.com/pages/global.php?page=telechargement}.

\par Une implémentation de référence uniquement en PHP, et contenant que peut de fonctionnalités par rapport à ce
que dit le PDF peut aussi être obtenue, mais n'est guère utile excepté pour la vérification cryptographique de la
signature de la réponse. Mais encore, il faut arriver à traduire les fonctions propres à PHP, et ce n'est pas toujours
une mince affaire, mais fort heureusement, les algorithmes sont encore assez standards et l'on trouve vite de l'aide
quant à ces fonctions.

\par De plus, certaines informations concernants les numéros d'identification de marchand son incohérents
d'une documentation à l'autre, et le plus simple à ce niveau est encore de contacter le support.

\subsection{Accès concurrentiels}
\label{sub:acces_concurrentiels}
\par En production, le projet Sith tourne à l'aide d'uWSGI, qui s'occupe lui de gérer les différents processus du
logiciel. Cela se traduit par des accès concurrentiels à la base de donnée lors de l'appel de deux pages simultanément
qui ont besoin d'accèder aux mêmes ressources.

\par Le problème n'en est la plupart du temps pas un, mais il devient très critique lorsque la page appelée permet par
exemple de recharger un compte AE. Il ne faut alors surtout faire l'opération en double.

\par Pour protéger ces accès en double, on peut alors utiliser des transactions, et \textbf{Django} fournit une
abstraction très pratique: \verb-with transaction.atomic():-.

\par L'Eboutic, avec sa réponse de la banque, est très sujette à ces accès concurrents, et cela a posé quelques
problèmes dans les débuts. La plupart ont été résolu, mais il arrive encore dans les comptoirs d'avoir une vente en
double, sans pour autant avoir le débit du compte qui soit doublé. Cela ne pose pas foncièrement de problèmes, puisque
le solde du compte est tout de même valide, et c'est un problème très compliqué à debugger, puisqu'il survient très
rarement, mais il faudrait tout de même arriver à le résoudre un jour.


\chapter{Le SAS}
\label{sec:le_sas}
\par Développeur principal: Skia

\section{But}
\label{sub:but}
\par Fournir un système de galerie de photo:
\begin{itemize}
    \item Upload en ligne via un formulaire pour tous les cotisants.
    \item Modération pour l'équipe du SAS.
    \item Système d'identification des membres pour retrouver rapidement ses photos.
    \item Affichage des photos dans les différents album et sur la page "photo" du profil d'un utilisateur.
\end{itemize}

\section{Principaux problèmes}
\label{sec:principaux_problemes}

\subsection{Gestion des fichiers}
\label{sub:gestion_des_fichiers}
\par L'envoie en grande quantité de photos nécéssite une gestion des fichiers solide, en même temps qu'un formulaire
d'envoie efficace, capable d'envoyer plusieurs dizaines de photos en une seule action de l'utilisateur.

\par L'envoie est donc fait à l'aide de requêtes AJAX pour envoyer les photos une par une et éviter alors le timeout.

\par Concernant les fichiers une fois envoyé, ils sont en réalité traités par la classe \verb#SithFile# qui gère tous
les fichiers du site. Cela ne fait qu'une seule classe à développer, de même qu'un seul système de fichier avec une
seule arborescence, ce qui est beaucoup plus robuste.

\par La difficulté a aussi été de permettre le déplacement des fichiers, par couper-coller, tout en faisant de même dans
le système de fichier réel, afin d'avoir une arborescence cohérente même en cas de perte de la base de données.

\subsection{Optimisation des pages}
\label{sub:optimisation_des_pages}
\par La génération d'un grand nombre de requêtes SQL est un des principaux problèmes de ralentissement d'un site. Le
SAS, avec ses très nombreuses photos, qui requierent une validation des droits, a posé un gros problème à ce niveau.

\par Certaines pages ont pu mettre jusqu'à plus de 10 secondes à générer, ce qui est inconcevable pour une galerie de
photos, mais ce temps à pu être réduit à moins de 3 secondes.

\par L'astuce à été d'utiliser des actions utilisateurs, comme l'upload de nouvelles photos, pour faire plus de
traitement que nécessaire, afin de mettre en "cache" une grande partie des actions, comme par exemple la génération des
miniatures des albums.

\par Une autre technique pour gagner du temps est de mettre en cache certaines requêtes en forcant les \emph{QuerySet}
à s'évaluer dans une \emph{list} \textbf{Python} que l'on stocke afin d'obtenir sa longueur, au lieu de lancer d'abord
un \emph{count}, puis une itération des résultats, qui en utilisant directement l'ORM, conduit à réaliser deux
requêtes SQL.

\par Enfin, le passage à \textbf{HTTP/2} permettrait d'améliorer encore les performances côté utilisateur puisqu'il n'y
aurait plus qu'un seul \emph{socket} d'ouvert pour transférer toutes les photos d'une page par exemple, sans avoir
pour autant à toucher au code.


\chapter{Les élections}
\label{sec:les_elections}
\par Développeur principal: Sli

\section{But}
\label{sub:but}
\par Fournir un système d'élections:
\begin{itemize}
    \item Gestion des différentes élections comprenants à chaque fois une liste de postes pour lesquels les gens
        candidatent, ainsi qu'une gestion des listes, pour pouvoir classifier et répartir les candidatures.
    \item Gestion d'une page de vote, permettant aux gens autorisés de pouvoir voter.
    \item Affichage des résultats une fois le vote terminé.
    \item Pas compatible avec la législation française: trop contraignant et pas utile, puisque validation officiel en
        AG.
\end{itemize}

\section{Principaux problèmes}
\label{sec:principaux_problemes}

\subsection{Automatisation d'un widget particulier pour les formulaires}
\label{sub:automatisation_d_un_widget_particulier_pour_les_formulaires}
\par La demande est venue du \textbf{BdF} qui a voulu autoriser pour certains poste un nombre de vote supérieur à 1.
Cela signifie que l'on passe d'un choix simple, type \verb#radio# à un choix multiple, type \verb#checkbox#, tout cela
étant paramètrable dans l'élection.

\par Ce genre de choix n'étant pas disponible dans \textbf{Django} de base, il a fallut développer le \emph{widget} à
utiliser dans le formulaire qui permette cette configuration tout en validant bien les données reçues par rapport au
modèle, et éviter ainsi de pouvoir "tricher" en envoyant des requêtes erronées.

\subsection{Revue du code d'un autre développeur}
\label{sub:revue_du_code_d_un_autre_developpeur}
\par \emph{Sli} étant le principal développeur de cette application, un gros travail de revue de code a dû être
effectuer afin de garantir une certaine cohérence avec le reste du projet.

\par C'est un travail très long et fastidieux, car il faut bien revérifier chaque ligne, sur chaque fichier, tout en
faisant des commentaire lorsque quelque chose ne va pas. \textbf{Gitlab} a, à ce niveau, grandement facilité la tâche, à
l'aide de ses outils de \emph{merge request} assez avancés.

\par En plus du code en lui-même, il a fallut porter une attention particulière aux migrations. Ces fichiers générés
automatiquement par \textbf{Django} sont responsables du maintient d'une base de donnée cohérente malgré les évolutions
des modèles. Même si le code parait donc valide, il est impératif de surveiller que la chaîne de dépendance des dites
migrations ne soit pas cassée, au risque de problèmes potentiels au moment de la mise en production, ce qui entraînent à
coup sûr un \emph{downtime}.

\chapter{Les stocks}
\label{sub:les_stocks}
\par Développeur principal: Lo-J
\vskip 2em

\par Cette application s’occupe de la gestion des stocks des comptoirs de type « BAR ». Elle permet de suivre les
quantités restantes afin de pouvoir déterminer de manière automatisée quels sont les produits qu’il faut acheter et en
quelle quantité.

\section{Liste des modèles}
\label{sec:liste_des_modeles}

\subsection{Stock}
\par Un Stock possède un nom et est lié à la classe Counter. Ainsi, chaque Comptoir peut avoir son propre Stock.

\subsection{StockItem}
\par Un StockItem possède un nom, une quantité unitaire, une quantité effective, une quantité minimale et est lié à la
classe Stock ainsi qu’à la classe ProductType. De cette manière, chaque élément appartient à un Stock et il est
catégorisé de la même manière que les Products (qui sont les objets utilisés pour la vente dans les comptoirs).

\subsection{ShoppingList}
\par Une ShoppingList possède un nom, une date, un booléen (fait ou à faire), un commentaire et est liée à un Stock.
Chaque ShoppingList est donc liée à un Stock ce qui permet d’avoir des listes de courses spécifique à chaque comptoir.

\subsection{ShoppingListItem}
\par Un ShoppingListItem possède un nom, une quantité demandée, une quantité achetée et est lié à la classe StockItem, à
la classe ShoppingList et à la classe ProductType. Cela permet de pouvoir faire plusieurs listes de courses différentes
en même temps et d’en garder un historique.

\section{Fonctionnement}
\label{sec:fonctionnement}

\par Au départ, si le comptoir de type « BAR » n’a pas de stock, la seule chose qu’il est possible de faire est d’en
créer un. Ensuite, il va falloir créer les objets StockItem en indiquant pour chacun les quantités qu’il y a dans le
stock.

\par De plus, la personne (généralement le Responsable du lieu de vie) qui aura la responsabilité d’informatiser les
stocks devra aussi définir la quantité unitaire, effective et minimale.
\par Par exemple, les Cheeseburger vendus aux différents comptoirs sont achetés par boite de 6, la quantité unitaire
sera donc 6, la quantité effective correspondra au nombre de boites restantes dans le stock (c’est à dire dans la
réserve du lieu de vie, une boite sortie du stock est considérée comme consommée) et enfin, la quantité minimale servira
de valeur seuil.
\par Une fois l’état de la réserve retranscrit dans le site, il reste encore gérer les stocks de manière quotidienne.
Pour ce faire, l’application se décompose en 3 parties :
\begin{itemize}
    \item Création automatique des listes de courses
    \item Approvisionnement du stock
    \item Prise d’éléments dans le stock
\end{itemize}

\par Lorsque l’on accède à la partie qui s’occupe de la gestion des listes de courses, il y a un bouton permettant de
créer une liste de courses en fonction de l’état des stocks à cet instant, puis un premier tableau contenant les listes
de courses qu’il faut faire et enfin un second tableau servant d’historique des listes de courses déjà effectuées.
\par Pour chaque liste de course ainsi créée, qu’elle soit « faite » ou « à faire », il est possible de cliquer sur son
nom pour voir le détail de ce qu’elle comprend.

\subsection{Création automatique des listes de courses}
\par En cliquant sur le bouton permettant de créer une nouvelle liste de courses, il faut remplir un formulaire. Les
informations à donner dans ce formulaire sont le nom de la liste de course (par exemple, une liste spéciale pour
Leclerc), ensuite, apparaissent tous les StockItem ayant une quantité effective inférieure au seuil fixé par leur
quantité minimale. Il faut donc donner pour chacun de ces éléments une quantité à acheter. Enfin, un dernier champ de
commentaire peut être compléter, il sert à demander l’achat d’éléments qui n’apparaissent pas dans le Stock, par
exemple, des couteaux, fourchettes ou encore tasses...
\par Lors de la validation de ce formulaire, la liste de courses est créée et est ajoutée au tableau contenant les
listes de courses à faire.

\subsection{Approvisionnement du stock}
\par Au retour des courses, il faut ranger les produits achetés dans la réserve. À ce moment-là, il faut aussi mettre à
jour le stock. Une opération « Mettre à jour le stock » est disponible pour chaque liste de courses du tableau « À
faire ».
\par En effectuant cette action, il va falloir indiquer les quantités effectivement achetées. En effet, les quantités
demandées ne sont pas forcément celles achetées, c’est donc les quantités effectives qu’il faut ajouter au stock. Une
fois ce formulaire validé, la liste de couses passera de l’état « à faire » à l’état « faite ».

\subsection{Prise d’éléments dans le stock}
\par La réserve étant accessible aux barmen afin qu’ils puissent réapprovisionner les réfrigérateurs à tout moment,
l’interface permettant de prendre des éléments dans le stock a été ajoutée dans les onglets de l’interface des ventes
(là où le barman inscrit le code du compte du client qui souhaite commander quelque chose). Ainsi, en revenant de la
réserve, le barman doit indiquer le nombre de chaque produit qu’il a rapporté.
\par Dans un souci de simplicité pour le gérant du lieu de vie, ce formulaire de prise des éléments dans le stock est
aussi accessible depuis son interface de gestion.


\section{Améliorations à apporter}
\label{sec:amelioration_a_apporter}

\begin{itemize}
    \item Il n’est pas encore possible de modifier les quantités demandées pour un ou plusieurs des produits d’une liste
        de course. Il faudrait rendre cela possible car actuellement, il faut supprimer la liste de courses et la
        refaire en changeant les quantités souhaitées.
    \item Il faudrait améliorer la manière dont on ajoute les éléments non définis en tant que StockItem dans la liste
        de course. Un objet ShoppingListItem avec une référence « Null » vers la classe StockItem pourrait être créé à
        la place de compléter le champ de commentaires.
    \item Dans les améliorations sur le long terme, il faudrait que la décrémentation des quantités de chaque élément
        dans le stock soit automatique. En repensant une partie de l’architecture de l’application, on pourrait faire en
        sorte que chaque vente faite au comptoir face diminuer les quantités restantes (cela remplacerait le formulaire
        de « Prise d’éléments dans le stock », mais cela ne serait pas applicable à tous les produits mis en vente. Par
        exemple, pour les cacahuètes que nous vendons au bol et non par paquet)
    \item Avec le système de notifications qui a été mis en place sur le site, on pourrait faire en sorte que le ou les
        responsables des lieux de vie reçoivent une notification lorsque la liste de courses contient plus de 5 éléments
\end{itemize}


\chapter{La laverie}
\label{sec:la_laverie}
\par Développeur principal: Skia

\section{But}
\label{sub:but}
\par Cette application doit fournir un système de gestion de laverie. Cela comprend:
\begin{itemize}
    \item Un système de planning et de réservation de créneaux
    \item Un système de vente de jetons de laverie, lié aux comptoirs et au compte AE, permettant aux permanenciers de
        cliquer les jetons en même temps qu'ils vérifient l'état de la cotisation.
    \item Un système d'inventaire, pour gérer les différentes machines dans les différents lieux, et gérer également le
        retour des jetons après utilisation.
\end{itemize}

\section{Principaux problèmes}
\label{sec:principaux_problemes}

\subsection{Génération de plannings}
\label{sub:generation_de_plannings}
\par Il y a là beaucoup de cas à prendre en compte. Lorsque que quelqu'un veut réserver directement un "Lavage +
Séchage", un simple "Lavage", ou un simple "Séchage", il faut toujours vérifier la disponibilité des créneaux en
fonction du nombre de machine de chaque type présent dans la laverie en question \footnote{Belfort ou Sevenans, en
l'occurrence}, et cela représente vite un grand nombre de combinaisons à vérifier.

\par De plus, la réservation doit rester ergonomique, et s'afficher dans un format le plus lisible possible pour un
humain. Là dessus, un tableau est le plus approprié, avec chaque jour représenté par une colonne, et chaque créneau par
une ligne.\\
Mais cela ne représente malheureusement pas la temporalité, et la génération du tableau devient alors plutôt compliquée,
et d'autant plus si l'on veut qu'il soit sémantiquement correct en HTML.

\subsection{Gestion des timezones}
\label{sub:gestion_des_timezones}
\par La gestion et le stockage des crénaux implique l'utilisation de champs de type \verb#DateTime#. \textbf{Django} les
gère très bien, particulièrement au niveau des \emph{timezones}, où ce dernier n'hésite pas à lancer un warning lorsque
l'objet \emph{date} passé ne contient pas d'information de fuseau horaire.

\par Mais avec notre décalage d'une heure par rapport au temps UTC, tous les horaires se retrouvent décalés, et gérer
cela convenablement sans sortir d'avertissement a été plutôt compliqué. La solution a été de forcer un peu partout la
\emph{timezone} à UTC, afin de ne pas créer de décalage, mais en conservant tout de même l'information de fuseau
horaire, et sans tout casser lors du passage à l'heure d'hiver.


\chapter{La communication}
\label{sec:la_communication}
\par Développeur principal: Skia

\section{But}
\label{sub:but}
\par Cette application a plusieurs but:
\begin{itemize}
    \item Donner la possibilité au responsable communication d'éditer les différents textes, messages, et pages
        statiques du site.
    \item Fournir un système de news.
    \item Fournir un système de newsletter: le Weekmail.
\end{itemize}

\section{Principaux problèmes}
\label{sec:principaux_problemes}

\subsection{Envoie de mails}
\label{sub:envoie_de_mails}
\par Un outil de \emph{newsletter} nécessite l'envoie de mail. C'est là quelque chose de relativement compliqué à
tester, d'autant plus lorsqu'il s'agit de mailing-list contenant l'intégralité des étudiants de l'UTBM.

\par \textbf{Django} fournit toutefois un outil très pratique: il contient plusieurs \emph{backend} d'emails, dont entre
autre un \emph{SMTP}, et un \emph{console}. Le \emph{SMTP} est bien évidemment utilisé en production pour envoyer
effectivement les mails, mais il est compliqué à utiliser en développement, car il suppose que le développeur a à sa
disposition un serveur de ce type. On utilise alors le backend \emph{console}, qui affiche simplement dans le thread
d'execution de \textbf{Django} une version texte de l'email envoyé, avec d'une part les entêtes, d'autre part le corps
de message.

\par Mais autant pour tester l'envoie d'un mail unique à une adresse unique, cela fonctionne parfaitement bien, ce n'est
toutefois pas suffisant pour tester un envoie massif à plusieurs mailings, avec en plus encore d'autres adresses en
\emph{Bcc} pour les gens ne faisant pas partie des mailings "classiques", mais souhaitant quand même recevoir le
\textbf{Weekmail}.

\subsection{Amélioration de l'outil de recherche}
\label{sub:amelioration_de_l_outil_de_recherche}
\par Pour la gestion de l'AE, il est nécessaire de pouvoir rechercher et trouver efficacement n'importe quel membre, en
tapant au choix son nom, prénom, ou surnom, voire une combinaison de ces trois champs.

\par Mais une fonction de recherche aussi complexe est très difficile a mettre en place efficacement sans un traitement
préalable, d'où la nécessité d'indexer les entrées à chercher. Un indexeur étant très complexe, mais également très
courant, il n'a pas été difficile de trouver une application déjà existante fournissant ces fonctionnalités.

\par Le choix s'est porté sur \textbf{Haystack}, en l'utilisant avec l'indexeur \textbf{Whoosh}, plutôt efficace pour
des bases raisonnables, et surtout écrit en pure \textbf{Python}, donc ne nécessitant pas d'installation compliqué en
parallèle du site.

\par Le résultat est plutôt satisfaisant, mais il faudrait encore améliorer les résultats en utilisant les fonctions de
\emph{boost} pour certains champs. De plus, une certaine lenteur se fait encore sentir avec certaines recherches trop
communes ou générales.


\chapter{Conclusions personnelles}
\section{Skia}
\label{sec:skia}
\par Développer de nouvelles application m'a permis d'apréhender d'autres problématiques, comme la gestion des fichiers
dans le SAS, ou bien des contraintes de concurrence et d'atomicité sur l'Eboutic.

\par Mais la plus grosse partie de mon travail ce semestre a surtout été de superviser une équipe de développement
naissante, de relire les "Merge request", et de m'assurer de la cohérence du code des contributeurs avec le reste du
projet.

\par J'ai églament pu approfondir mon utilisation de Gitlab à travers ses outils de gestion de projet, de revue de code,
et de gestion des permissions sur les différentes branches.

\section{Lo-J}
\label{sec:lo_j}
\par Je suis très heureux d’avoir pu participer à ce projet de TO52 sur le développement de modules sur le site de l’Association des Étudiants. J’ai pu apprendre à travailler avec un nouvel environnement informatique tout en contribuant au développement d’outils pour l’association dont je suis Président, le Bureau des Festivités.

\subsection{Django}
\par Ayant déjà travaillé avec le framework Spring et Java durant mon stage ST40, j’ai pu m’appuyer sur des notions générales afin d’apprendre et de comprendre le fonctionnement du Python et de Django que je ne connaissais pas du tout.

\par Mon apprentissage a été assez long au départ, car il y avait beaucoup d’informations à intégrer. Django est un framework très pratique qui permet d’effectuer de nombreuses tâches assez rébarbatives de manière automatique certes, mais encore faut-il comprendre ce qu’il se passe en arrière-plan. C’est cet apprentissage qui m’a pris le plus de temps.

\par Mon deuxième point de difficulté a été les formulaires. Là encore, Django est très pratique dès lors qu’il s’agit de faire un formulaire avec tous les champs d’un même Model. Cependant, il m’a fallu de l’aide et du temps pour comprendre comment faire pour ajouter d’autres champs en plus de ceux du Model au formulaire et pour comprendre comment récupérer les valeurs associées à chacun d’eux.

\subsection{Git}
\par J’ai eu aussi à apprendre le fonctionnement de Git que j’avais déjà pu manipuler quelque peu mais il me manquait quand même beaucoup d’éléments.

\par Aujourd’hui, je pense pouvoir dire que j’ai progressé dans ce domaine mais il me reste encore bien des choses à apprendre pour être capable de l’utiliser de manière efficace.


\end{document}

