%%
%
% Skia
% skia@libskia.so
%
%%

\documentclass[a4paper]{report}

%packages
\usepackage[utf8]{inputenc}
\usepackage[francais]{babel}
\usepackage{graphicx}\graphicspath{{pix/}}
\usepackage{float}
\usepackage{scrextend}
\usepackage[T1]{fontenc}
\usepackage{color}
\usepackage{fancyhdr}
%Options: Sonny, Lenny, Glenn, Conny, Rejne, Bjarne, Bjornstrup
\usepackage[Bjornstrup]{fncychap}
\usepackage{minted}
\usepackage[colorlinks=true,linkcolor=black]{hyperref}
\usepackage{pdfpages}
%\usepackage{titlesec, blindtext, color}

%pdf metadata
\hypersetup{
    unicode=true,
    colorlinks=true,
    citecolor=black,
    filecolor=black,
    linkcolor=black,
    urlcolor=black,
    pdfauthor={Skia <skia@libskia.so>},
    pdftitle={},
    pdfcreator={pdftex},
    pdfsubject={},
    pdfkeywords={},
}


\definecolor{keywords}{RGB}{200,0,90}
\definecolor{comments}{RGB}{50,50,253}
\definecolor{red}{RGB}{160,0,0}
\definecolor{brown}{RGB}{160,100,100}
\definecolor{green}{RGB}{0,200,0}
\definecolor{darkgreen}{RGB}{0,130,0}
\definecolor{gray}{RGB}{100,100,100}


%inner meta
\title{Sith: Détail de quelques applications}
\author{Skia (Florent JACQUET)\\
Lo-J (Guillaume Renaud)}
\date{Dernière version: \today}

\begin{document}

\maketitle

\tableofcontents

\chapter{Introduction}

\chapter{Eboutic}
\label{sec:eboutic}
\par Développeur principal: Skia

\section{But}
\label{sub:but}
\par Fournir une boutique en ligne, avec paiement sécurisé, compatible avec l'API de paiement du Crédit Agricole.
\begin{itemize}
    \item Gérer les cotisations
    \item Gérer les rechargements de compte AE
    \item Gérer différents groupes de vente
\end{itemize}

\section{Principaux problèmes}
\label{sec:principaux_problemes}

\subsection{Interaction avec l'API}
\label{sub:interaction_avec_l_api}

\subsection{Accès concurrentiels}
\label{sub:acces_concurrentiels}



\chapter{Le SAS}
\label{sec:le_sas}
\par Développeur principal: Skia

\section{But}
\label{sub:but}
\par Fournir un système de galerie de photo:
\begin{itemize}
    \item Upload en ligne via un formulaire pour tous les cotisants.
    \item Modération pour l'équipe du SAS.
    \item Système d'identification des membres pour retrouver rapidement ses photos.
    \item Affichage des photos dans les différents album et sur la page "photo" du profil d'un utilisateur.
\end{itemize}

\section{Principaux problèmes}
\label{sec:principaux_problemes}

\subsection{Gestion des fichiers}
\label{sub:gestion_des_fichiers}

\subsection{Optimisation des pages}
\label{sub:optimisation_des_pages}


\chapter{Les élections}
\label{sec:les_elections}
\par Développeur principal: Sli

\section{But}
\label{sub:but}
\par Fournir un système d'élections:
\begin{itemize}
    \item Gestion des différentes élections comprenants à chaque fois une liste de postes pour lesquels les gens
        candidatent, ainsi qu'une gestion des listes, pour pouvoir classifier et répartir les candidatures.
    \item Gestion d'une page de vote, permettant aux gens autorisés de pouvoir voter.
    \item Affichage des résultats une fois le vote terminé.
    \item Pas compatible avec la législation française: trop contraignant et pas utile, puisque validation officiel en
        AG.
\end{itemize}

\section{Principaux problèmes}
\label{sec:principaux_problemes}

\subsection{Automatisation d'un widget particulier pour les formulaires}
\label{sub:automatisation_d_un_widget_particulier_pour_les_formulaires}

\subsection{Revue du code d'un autre développeur}
\label{sub:revue_du_code_d_un_autre_developpeur}


\chapter{Les stocks}
\label{sub:les_stocks}
\par Développeur principal: Lo-J

\section{But}
\label{sec:but}

\section{Principaux problèmes}
\label{sec:principaux_problemes}


\chapter{La laverie}
\label{sec:la_laverie}
\par Développeur principal: Skia

\section{But}
\label{sub:but}
\par Cette application doit fournir un système de gestion de laverie. Cela comprend:
\begin{itemize}
    \item Un système de planning et de réservation de créneaux
    \item Un système de vente de jetons de laverie, lié aux comptoirs et au compte AE, permettant aux permanenciers de
        cliquer les jetons en même temps qu'ils vérifient l'état de la cotisation.
    \item Un système d'inventaire, pour gérer les différentes machines dans les différents lieux, et gérer également le
        retour des jetons après utilisation.
\end{itemize}

\section{Principaux problèmes}
\label{sec:principaux_problemes}

\subsection{Génération de plannings}
\label{sub:generation_de_plannings}

\subsection{Gestion des timezones}
\label{sub:gestion_des_timezones}


\chapter{La communication}
\label{sec:la_communication}
\par Développeur principal: Skia

\section{But}
\label{sub:but}
\par Cette application a plusieurs but:
\begin{itemize}
    \item Donner la possibilité au responsable communication d'éditer les différents textes, messages, et pages
        statiques du site.
    \item Fournir un système de news.
    \item Fournir un système de newsletter: le Weekmail.
\end{itemize}

\section{Principaux problèmes}
\label{sec:principaux_problemes}

\subsection{Envoie de mails}
\label{sub:envoie_de_mails}

\subsection{Amélioration de l'outil de recherche}
\label{sub:amelioration_de_l_outil_de_recherche}


\chapter{Conclusions personnelles}
\section{Skia}
\label{sec:skia}
\par Développer de nouvelles application m'a permis d'apréhender d'autres problématiques, comme la gestion des fichiers
dans le SAS, ou bien des contraintes de concurrence et d'atomicité sur l'Eboutic.

\par Mais la plus grosse partie de mon travail ce semestre a surtout été de superviser une équipe de développement
naissante, de relire les "Merge request", et de m'assurer de la cohérence du code des contributeurs avec le reste du
projet.

\par J'ai églament pu approfondir mon utilisation de Gitlab à travers ses outils de gestion de projet, de revue de code,
et de gestion des permissions sur les différentes branches.

\section{Lo-J}
\label{sec:lo_j}



\end{document}

